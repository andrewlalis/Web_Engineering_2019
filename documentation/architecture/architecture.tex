\documentclass{article}
\usepackage[utf8]{inputenc}
\usepackage{listings}
\usepackage{fancyvrb}
\usepackage{fontspec}
\setmainfont{Olde English}

\fvset{tabsize=2}

\lstset{
    breaklines=true,
    tabsize=2,
    basicstyle=\small
}

\title{Architecture \\ \small{Web Engineering, 2019}}
\author{Andrew Lalis, Project Leader and Framework Specialist \\ 
George Rakshiev, Design and Requirements Engineer \\ 
Tom den Boon, Full Stack Software Superhero}

\begin{document}

\maketitle

\section{Introduction}
	While some information regarding the architectural choices of this project will be discussed in this document, to get a more in-depth explanation of the Andypoints framework which is used exclusively in this project, please consult the readme file available in the 'sources' directory.
	
	Note that currently, this document is a work-in-progress as the technology we use evolves to fit the requirements of the project.
	
\section{Technology Choices}
	Because both Tom and Andrew have some working knowledge of PHP, this was chosen as the programming language with which to design the API. It was decided that no pre-existing framework would be used, and instead an in-house solution would be created which is minimalistic and does just enough to get the job done, without adding unnecessary bloating to the project.
	
	George is, due to medical reasons, unable to program in PHP, and so is the designated design specialist, and will dedicate his resources to developing a front-end representation of the project when the time comes. For such a front-end representation, it is anticipated that the Andypoints Framework will be extended to include a very rudimentary templating engine that can parse template files into raw PHP, to give as much speed as possible.

\end{document}
